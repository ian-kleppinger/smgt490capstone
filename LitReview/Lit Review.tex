\documentclass[12pt]{article}

% ---------- Page layout ----------
\usepackage[margin=1in]{geometry}

\usepackage[style=apa, backend=biber]{biblatex}
\addbibresource{LitReview/sources.bib} % your .bib file

\title{Literature Review for SMGT 490}

\author{
  Ian Kleppinger \\
  Department of Sport Management, Rice University\\
  \texttt{imk5@rice.edu}
}
\date{January 25, 2026}

\begin{document}
\maketitle

This paper is a review of the relevant literature for my senior capstone project as a Sport Analytics major in the Department of Sport Management at Rice University. My research question (as it currently stands) is: 
\begin{center}
\begin{minipage}{0.75\textwidth}
\centering
\itshape
\textbf{How much do attack angle and its related metrics (attack direction, swing path tilt, etc.) vary at the league, team, and player level? How should players be making measurable changes to their swings in certain situations, and how much of that are they or are they not doing in the status quo?}
\end{minipage}
\end{center}

The data on attack angle is new, having only been publicly released by MLB Statcast during the 2025 season \parencite{petriello2025statcast}. This was particularly groundbreaking for two reasons. First, the concept of measuring this data was very new to baseball, and had only been done by anyone in any capacity for a few years. Second, the data publicly available or easily collected by researchers was largely done with small sensors on the ends of the bat, rather than with multi-camera high-speed motion capture setups, which cost several thousands of dollars. A study done at Mississippi State University found that the differences between these two methods of measurement were statistically significant. The bat handle sensor had a tendency to overestimate both bat speed and vertical angle, while underestimating the attack angle when compared to a motion capture system \parencite{stewart2021validation}. The consistent inaccuracy in data from bat handle sensors made it challenging to do research on the impact of small variations in attack angle and related metrics before the publication of the MLB stadium data from Statcast, which is all done with motion capture systems. 

Because of the novelty of this data, there isn't a lot of published research exploring attack angle and its related metrics. However, while the attack angle data is new as of 2025, bat tracking data as a whole is not. Bat speed and swing length have been publicly available for well over a year at this point \parencite{passan2024mlbstatcast}. A number of papers have been published investigating the characteristics of those data, and some even serve as an inspiration for the work done in this paper.

A 2024 FanGraphs article looked at correlations between bat speed and holistic offensive statistics, like Weighted Runs Created Plus (wRC+) to test the hypothesis that higher bat speed is always best for maximizing run scoring or offensive success \parencite{clemens2024statcastbat}. wRC+ is a stat that matches each plate appearance (PA) outcome to its run value, rather than treat or weight them all in the same way like in more traditional stats such as Batting Average (AVG) and On Base Percentage (OBP) \parencite{mlb2025wrcplus}. The article found that there "isn't much correlation" between bat speed and overall offensive production, citing successful hitters with lower bat speeds such as Luis Arraez, Steven Kwan, Jose Altuve, and Marcus Semien \parencite{clemens2024statcastbat}. Many baseball fans will point out that these hitters all get at least part of their success from a contact-oriented approach (rather than a power approach), which makes from a fundamental physics perspective since swinging the bat faster will make the ball go faster/farther, all other things held equal (most notably quality of contact). The obvious takeaway here is that there's a lot more to swinging a bat and hitting a baseball than bat speed, although it is undeniably a part of the equation.

Researchers have looked into this power/contact tradeoff as it pertains to bat speed. A study published in July of 2025 looked at this data and fit models the estimate how bat speed and swing length impact contact and power outcomes. They modeled the tradeoff between contact and power and found that batters can make more contact and strikeout less if they reduce their bat speed, but on average the positive impact on offensive production is counteracted by the loss of power \parencite{powers_yurko_2025_swinging}. This corroborates the findings from \cite{clemens2024statcastbat} about hitters who are successful with widely varying bat speeds as a result of different plate approaches. 

However, certain swing characteristics can give the hitter more margin of error in their timing, which effectively makes hitting easier \parencite{nakashima2025timingerror}. They found that the "acceptable range of timing errors," or how much hitter timing can be off and still make sweet spot contact, is highest when the angle of the bat relative to the flight path of the ball is closest to 0\textdegree. This is most true when the angles are aligned in both the vertical and horizontal plane. If these two angles are perfectly aligned, the player can theoretically hit the ball with the bat's sweet spot on pitches up to 10 km/h different in speed by using the exact same swing \parencite{nakashima2025timingerror}. It will be interesting to look at angles in this analysis and see how hitters with historically different approaches (contact vs power) vary in attack angle. According to prior research, it would make sense if the contact hitters had less variance relative to the ball's flight path, as this would maximize the margin of error on their timing.

A study in Japan did similar research into opposite-field hitting (hitting the ball to right field for right handed batters) and found that hitters can change the vertical angle of their swing to hit the ball more consistently towards right field \parencite{shimura2018optimal}. Hitting lower on the baseball (i.e. "under-cutting" the baseball) can make up for the angle of the bat not being pointed towards the opposite field if the goal is to hit it that direction. There is also, however, the issue of under-cutting the ball too much, which would cause it be hit softly and/or up in the air, both of which often negate any benefit from hitting it in the intended direction (with the exception of potentially sacrificing a runner from second to third). In general, changing the angles (both vertical and horizontal) in which a hitter impacts the ball can get it headed more towards the opposite field, in addition to the traditional wisdom of swinging later or "letting the ball get deep." One area I plan to explore in my work pertaining to attack angle is looking at situations where the conventional approach or old-school baseball thought is to "hit behind the runner," which is the opposite field for a right-handed batter. If there is a runner on 1st or 2nd base, hitting "behind" them often allows them a better chance of being able to safely advance, and it will be interesting to see if players are adjusting their swings to do this, or if they could more successfully do so by making adjustments. 

\printbibliography[title={References}]


\end{document}